% LaTeX resume using res.cls
\documentclass[10pt,centered]{./res} %use line for small header
%\usepackage{helvetica} % uses helvetica postscript font (download helvetica.sty)
%\usepackage{newcent}   % uses new century schoolbook postscript font
\setlength{\topmargin}{-.8in}  % Start text higher on the page
\setlength{\textheight}{10.2in}  % increase textheight to fit more on a page
\setlength{\headsep}{0.2in}     % space between header and text
\setlength{\headheight}{12pt}   % make room for header
\usepackage{fancyhdr}  % use fancyhdr package to get 2-line header
\usepackage{hyperref}
\renewcommand{\headrulewidth}{0.5pt} % suppress line drawn by default by fancyhdr
\fancyheadoffset[L]{\sectionwidth}
% \lhead{\hspace*{-\sectionwidth}Johannes S. Otterbach} % force lhead all the way left
\lhead{Johannes S. Otterbach} % force lhead all the way left
\rhead{Page \thepage}  % put page number at right
\cfoot{}  % the footer is empty
\pagestyle{fancy} % set pagestyle for the document

\usepackage{color}

\definecolor{ResumeBlue}{RGB}{56, 115, 179}

\usepackage{array}

\begin{document}
\thispagestyle{empty} % this page does not have a header

% Center the name over the entire width of resume:
\moveleft.5\hoffset\centerline{\Large\bf Dr. Johannes S. Otterbach}
% Draw a horizontal line the whole width of resume:
\moveleft\hoffset\vbox{\hrule width\resumewidth height 1pt}\smallskip
% address begins here
% Again, the address lines must be centered over entire width of resume:
\moveleft.5\hoffset\centerline{johannesotterbach@gmail.com $\vspace*{.2cm}\bullet\vspace*{.2cm}$ +49 1520 3437776}
\vspace{-.3cm}
\moveleft.5\hoffset\centerline{linkedin.com/in/jotterbach $\vspace*{.2cm}\bullet\vspace*{.2cm}$ jotterbach.github.io $\vspace*{.2cm}\bullet\vspace*{.2cm}$ github.com/jotterbach}

\vspace*{-1.5cm}
\begin{resume}
\vspace{0.4cm}

% \section{\color{ResumeBlue}PROFILE}
% Self-motivated Ph.D.-level Physicist with a curious, analytical mind and a passion for all things AI, quantum \& data. Experience managing and analyzing data using Python (NumPy, SciPy, pandas, scikit-learn), Apache Spark (SparkSQL, MLlib), TensorFlow, Postgres, MATLAB, {\sc Mathematica} and developing algorithms and software for near-term quantum hardware. Extensive experience with advanced mathematics, statistics and applied machine learning, as well as presenting and visualizing complex concepts to diverse audiences.

\section{\color{ResumeBlue}PROFESSIONAL DEVELOPMENT}
{\bf VP of Machine Learning Research} \hfill04/2021 - present \\
{\it Merantix Labs, Berlin, Germany}
% \begin{itemize}
%   \item Growing and leading the research division through mentoring, coaching, active code development.
%   \item Support of research project managment through high-level direction setting and conceptualization
% \end{itemize}

{\bf Scientific AI Advisor to Syntegra.io } \hfill10/2019 - present \\
{\it Syntegra.io, San Francisco, CA}

{\bf Machine Learning Researcher} \hfill06/2018 - 02/2021 \\
{\it OpenAI, San Francisco, CA}
% \begin{itemize}
%   \item Basic research in the area of Unsupervised and Generative Models with focus on Energy-Based Models and Normalizing Flows.
%   \item Contributed to the rollout to the OpenAI API by close iteration with launch dedicated partners.
%   \item Investigated the generalization of GPT-3 for multi-language support.
% \end{itemize}

{\bf Research Scientist and Software Engineer} \hfill04/2017 - 05/2018 \\
{\it Rigetti Quantum Computing, Berkeley, CA}
% \begin{itemize}
%   \item Prototyping and demonstrating applications for near-term quantum devices, such as Quantum Machine Learning and Combinatorial Optimization Problems.
%   \item Developing and maintaining an OCaml-based simulator of a quantum processing unit.
%   \item Managing, coordinating and actively participating in a small research team for near-term applications.
%   \item Engaging with customers; estimating benefits using quantum computations and translating problems to quantum algorithms.
% \end{itemize}

{\bf Senior Data Scientist} \hfill12/2016 - 03/2017 \\
{\bf Data Scientist} \hfill08/2015 - 11/2016 \\
{\it LendUp, San Francisco, CA}
% \begin{itemize}
%   \item Architect of new machine learning model scoring service with ability to serve models developed in several different languages and frameworks.
%   \item Implemented Python variants of various learning algorithms, such as Generalized Additive Models and Constrained Linear Models.
%   \item Contributed to key algorithms to generate model insights and auditability for regulatory compliance.
%   \item Supported Data Scientists with ad-hoc and production algorithms for feature analysis and selection. Provided dashboards and automated reports for business stakeholders.
%   \item Developed and deployed several models for credit underwriting, including models for new products.
%   \item Analysed and integrated new data sources into production systems to increase data redundancy.
% \end{itemize}

{\bf Infrastructure Quality Engineer} (Machine Learning)\hfill4/2014 - 7/2015 \\
{\it Palantir Technologies, London, UK (until 1/2015) and Palo Alto, CA}
% \begin{itemize}
%   \item Analyzed TB-sized, disparate customer-dataset and implemented new propensity model pipeline using Apache Spark, surfacing previously unknown churn indicators.
%   \item Solidified and scaled end-to-end PySpark ETL-machine learning pipeline, resulting in a $\sim$5x increase in handled data-scale and $\sim$5x decrease of training time.
%   \item Reduced feature engineering development times by 3x through creating new featurization prototypes in quick iterations with product and data-science teams.
%   \item Deployed, debugged and maintained complex, distributed software stacks, containing Apache Spark, Hadoop HDFS and IPython Notebook servers, on cloud-based AWS systems. Optimized the stacks for best computational performance and stability.
%   \item Developed CometD-based user-scale testing and analytics framework resulting in a $\sim$10x improvement in handled users.
% \end{itemize}

{\bf Postdoctoral Research Fellow} (Theoretical Quantum Physics)\hfill 9/2011 - 3/2014 \\
{\it Harvard Quantum Optics Center, Cambridge, MA}
% \begin{itemize}
%   \item Studied phase diagrams of strongly interaction 1D cold atom systems with numeric and analytic tools.
%   \item Simulated the time-evolution of models with spatial and temporal randomness using Markov processes and ensemble theory, creating insights into highly correlated states of matter.
%   \item Explained and matched experimental observations to theoretical models using fitted statistical simulations and analytic solutions.
%   \item Presented research results to general as well as expert audiences through invited seminars, conferences, talks and posters.
%   \item Collaborated, influenced and contributed to research projects with international teams.
% \end{itemize}

\section{\color{ResumeBlue}EDUCATION}

{\bf Ph.D. in Physics}, {\it GPA: 4.0 with distinction}, 10/2011
\begin{itemize}
 \item[] Theoretical Quantum Optics Group of Prof. Dr. M. Fleischhauer\\ University of Kaiserslautern, Germany
\end{itemize}
\vspace*{-0.2cm}
{\bf MSc. (Diploma) in Physics}, {\it GPA: 3.93}, 5/2008
\begin{itemize}
 \item[] University of Kaiserslautern, Germany
\end{itemize}

% \section{\color{ResumeBlue}TECHNICAL SKILLS}

% \vspace*{0.2cm}
% \hspace*{-.9cm}
% \begin{tabular}{p{5.85in}>{\raggedleft\arraybackslash}p{.1in}}
% \begin{itemize}
%   \item Programming languages: Python, Java, Apache Spark, Scala, JavaScript, SQL and Shell scripting. Familiarity with OCaml, Cyton/C, Hadoop HDFS, AWS S3, R as well as ReactJS, Redux and Gatsby.
%   \item Experience with mathematical and statistical Python libraries such as pandas, scikit-learn, NumPy and SciPy, PyTorch, TensorFlow, Owl, and software such as MATLAB and {\sc Mathematica}.
%   \item Advanced mathematics and physics toolset paired knowledge of software best practices and applied machine learning ideally suited to tackle bleeding-edge challenges in AI and Deep Learning.
% \end{itemize}
%  & \\
% \end{tabular}

% \vspace*{-0.2cm}
\section{\color{ResumeBlue}SCHOLARSHIPS AND AWARDS}
\vspace*{0.2cm}
\hspace*{-.3cm}
\begin{tabular}{p{5.0in}>{\raggedleft\arraybackslash}p{.85in}}
  {OpenAI Fellowship} 
 &  2018 \\
  {Prize Fellowship} of the Harvard Quantum Optics Center
 &  2011-2013 \\
  {2011 Award} of the Friends of the University of Kaiserslautern for an outstanding scientific performance as a Ph.D. student in physics
 & 2012 \\
 {2009 Young Talent Award} (Nachwuchspreis) of the Department of Physics of the TU Kaiserslautern for an outstanding MSc thesis& 2009 \\
  {Foundation of German Business} scholarship
 & 2005-2008 \\
\end{tabular}

\section{\color{ResumeBlue}LANGUAGE SKILLS}
\vspace*{0.2cm}
German: Native speaker. English: Fluent. Swedish and French: Basic

\pagebreak
\section{\color{ResumeBlue}PUBLICATION LIST}
Also see: \url{https://scholar.google.com/citations?user=yZS4ce8AAAAJ&hl=en&authuser=1}
\vspace*{0.2cm}
\begin{enumerate}
  \item K. Ditschuneit, \& J.Otterbach, \textit{Auto-Compressing Subset Pruning for Semantic Image Segmentation}, arXiv:2201.11103

  \item D. Sreenivasaiah, J. Otterbach \& T. Wollmann, \textit{
    MEAL: Manifold Embedding-based Active Learning}, 2021 IEEE/CVF International Conference on Computer Vision Workshops (ICCVW), 2021, pp. 1029-1037, doi: 10.1109/ICCVW54120.2021.00120.
  
  \item S. v. Baußnern$^\dagger$, J. Otterbach$^\dagger$, A. Loy, M. Salzmann \& T. Wollmann, \textit{DAAIN: Detection of Anomalous and Adversarial Input using Normalizing Flows}, arxiv:2105.14638
  
  \item J. Otterbach \& T. Wollmann, \textit{Chameleon: A Semi-AutoML framework targeting quick and scalable development and deployment of production-ready ML systems for SMEs}, arxiv:2105.03669 (Accepted at KI-KMU 2021)
  
  \item J. Otterbach, J. Ward, M. P. da Silva, N. C. Rubin,\textit{Selecting parameters for a quantum approximate optimization algorithm (QAOA)}, Patent number: 10846366 (USA).
  
  \item C. M. Wilson, J. Otterbach \& Rigetti Computing, \textit{Quantum Kitchen Sinks: An algorithm for machine learning on near-term quantum computers}, arxiv:1806.08321
  
  \item S. Caldwell \& Rigetti Computing \textit{Parametrically-Activated Entangling Gates Using Transmon Qubits}, Physical Review Applied 10 (3), 034050 (2018).

  \item M. Reagor \& Rigetti Computing, \textit{Demonstration of Universal Parametric Entangling Gates on a Multi-Qubit Lattice}, Science Advances, 4, eaao3603 (2018)
  
  \item J. Otterbach \& Rigetti Computing, \textit{Unsupervised Machine Learning on a Hybrid Quantum Computer}, arxiv:1712.05771
  
  \item Q. Wang, J. Otterbach, S. F. Yelin \textit{Interacting in-plane molecular dipoles in a zig-zag chain}, Phys. Rev. A 96, 043615 (2017)
  
  \item J. Otterbach \& M. Lemeshko, \textit{Dissipative Preparation of Spatial Order in Rydberg-Dressed Bose-Einstein Condensates}, Phys. Rev. Lett. 113, 070401 (2014).
  
  \item F. Bariani, J. Otterbach, H. Tan, P. Meystre, \textit{Single-atom quantum control of macroscopic mechanical oscillators}, Phys. Rev. A 89, 011801(R) (2014).
  
  \item J. Otterbach, M. Moos, D. Muth, M. Fleischhauer, \textit{Wigner Crystallization of Single Photons in Cold Rydberg Ensembles}, Phys. Rev. Lett. 111, 113001 (2013).
  
  \item E. G. Dalla Torre, J. Otterbach, E. Demler, V. Vuletic, M. D. Lukin, \textit{Dissipative Preparation of Spin Squeezed Atomic Ensembles in a Steady State},, Phys. Rev. Lett. 110, 120402 (2013).
  
  \item S. D. Bennett, N. Y. Yao, J. Otterbach, P. Zoller, P. Rabl, M. D. Lukin, \textit{Phonon-induced spin-spin interactions in diamond nanostructures: application to spin squeezing}, Phys. Rev. Lett. 110, 156402 (2013).
  
  \item M. J. Edmonds, J. Otterbach, R. G. Unanyan, M. Fleischhauer, M. Titov, P. Öhberg, \textit{From Anderson to anomalous localization in cold atomic gases with effective spin-orbit coupling}, New J. Phys. 14, 073056 (2012).
  
  \item J. Ruseckas, V. Kudriasov, G. Juzeliunas, R. G. Unanyan, J. Otterbach, M. Fleischhauer, \textit{Photonic band-gap properties for two-component slow light}, Phys. Rev. A 83, 063811 (2011).
  
  \item A. V. Gorshkov, J. Otterbach, M. Fleischhauer, T. Pohl, M. D. Lukin, \textit{Photon-Photon Interactions via Rydberg Blockade}, Phys. Rev. Lett. 107, 133602 (2011).
  
  \item D. Petrosyan, J. Otterbach, and M. Fleischhauer, \textit{Electromagnetically induced transparency with Rydberg atoms}, Phys. Rev. Lett. 107, 213601 (2011).
  
  \item J. Otterbach, J. Ruseckas, R. G. Unanyan, G. Juzeliunas, and M. Fleischhauer, \textit{Effective magnetic fields for stationary light}, Phys. Rev. Lett. 104, 033903 (2010).
  
  \item A. V. Gorshkov, J. Otterbach, E. Demler, M. Fleischhauer, and M. D. Lukin, \textit{Photonic Phase Gate via an Exchange of Fermionic Spin Waves in a Spin Chain}, Phys. Rev. Lett. 105, 060502 (2010).
  
  \item R. G. Unanyan, J. Otterbach, M. Fleischhauer, J. Ruseckas, V. Kudriasov, and G. Juzeliunas, \textit{Spinor Slow-Light and Dirac particles with variable mass}, Phys. Rev. Lett. 105, 173603 (2010).
  
  \item J. Otterbach, R. G. Unanyan, M. Fleischhauer, \textit{Confining stationary light: Dirac dynamics and Klein tunneling}, Phys. Rev. Lett. 102, 063602 (2009).
  
  \item R. G. Unanyan, J. Otterbach, M. Fleischhauer, \textit{Confinement Limit of Dirac particles in scalar 1D potentials}, Phys. Rev. A 79, 044101 (2009).
  
  \item F. E. Zimmer, J. Otterbach, R. G. Unanyan, B. W. Shore, M. Fleischhauer, \textit{Dark-State Polaritons for multi-component and stationary light fields}, Phys. Rev. A 77, 063823 (2008).
  
  \item M. Fleischhauer, J. Otterbach, R. G. Unanyan, \textit{Bose-Einstein condensation of stationary-light polaritons}, Phys. Rev. Lett. 101, 163601 (2008).

\end{enumerate}



% \section{\color{ResumeBlue}ACTIVITIES}
% \vspace*{0.2cm}
% Avid boulderer and climber. Enjoys slacklining and a good game of Ultimate Frisbee with friends. Good food or an outdoor trip are always welcome.

\end{resume}
\end{document}
